\section{Background}
\label{BabyGotBack}
\begin{comment}
A common way to increase computer performance, is to increase the capacity for 
parallelism. For practical usage, however, this is difficult to implement, due 
to low-level GPU-specific languages requiring domain specific knowledge to make 
full use of that capacity. A vast amount of work has gone into transforming 
high-level hardware-agnostic code into these low-level GPU-specific languages 
\cite{inc-flat}.  \end{comment}
In this section, we want to present the theory and concepts necessary, to
understand how to autotune Futhark programs. In order to grasp these concepts,
a simplified explanation of the hardware we execute parallel code on,
that being GPUs, is in order. 

The idea of a GPU is to devote more transistors to compute, than a CPU, with
the consequence of less transistors for data caching and flow control. More
specifically, a GPU contains a high number of \textit{streaming
processors (SPs)}. SPs are general purpose, but have significant
limitations, like not supporting branch prediction, however this is made up for 
by
instruction-level parallelism. A thread is an abstraction of a single execution 
of
some function. An SP is capable of executing a large number of threads at once 
while being
able to efficiently switch context. We say that on an SP, there can fit one 
workgroup,
so if an SP can execute 128 threads, we say that SP (and therefore that GPU) 
has a
maximum workgroup size of 128. Since all the threads in a workgroup are running
on the same SP they are able to communicate through the SPs
internal memory, called \textit{shared memory}, which is much faster than
having to use the GPUs ''main'' memory, called \textit{global memory}.
Because using the shared memory is fast, while using the global memory is
slow, splitting work across SPs (workgroups) and handle their intermediate
results can be expensive.

\subsection{Parallelism} In the real world there are limitations to parallel 
hardware. We
commonly execute parallel work on GPUs, due to its high number of threads. The
limitation on GPUs, most essential to this report, is that a thread, may not
spawn additional threads. \footnote{While this is not strictly true, in 
practice it remains
so.} We say that since a GPU thread cannot spawn additional threads, it only
supports \textit{flat} parallelism. In the following sections approaches to 
work within this limitation are explored.


\subsection{Flat parallelism}
Many problems do not have a flat structure, as our current platform
effectively requires. Imagine a function (\texttt{drawline}) that, given two
endpoints, will determine all the pixel that lie in a straight line in between.
This function will have a flat structure, and can therefore be executed
in parallel. The function cannot itself be drawn in parallel as this would be
nested parallelism, which is not supported. So while a user would be able to draw a
single line in parallel with this function, it would not be possible to use it
to draw multiple lines in parallel.

Therefore we need to map nested parallelism to a flat structure. This problem
has already been conceptually solved by \textit{flattening}, and implemented in
the language NESL \cite{nesl}. 

When flattening a datastructure, we want to keep the information regarding the
original nest. In the original work presenting flattening \cite{flat} this is 
done
with a structure called \textit{field}. A \textit{field} has the form of 
\texttt{(nest
information, value)}, where a value can be an additional field, allowing for
arbitrary depth. For a multidimensional array the field would be
\texttt{(segment lengths, values/fields)}, below here is a concrete example.  
\begin{center}
  \texttt{[([1,2,3], [4,5], [5,7,8,9])] $\to$ ([3], ([3,2,4],
[1,2,3,4,5,6,7,8,9]))}
\end{center}
The length of the outer array is 3, and since there is only a single segment
that is the only lenght we save. In the inner \textit{field} we need to save
the length of all three of the segments, and all the values of the original
subarrays.\\

Flattening allows for completely flat data that can be mapped directly to 
the
GPU. But, as we saw in the example above, flattening increases the memory cost and
it does not take communication cost into account. This is a problem as GPUs are 
not
infinitely parallel, they have some maximum physical capacity for parallelism.
Going beyond that capacity will not result in futher computational gain and the
flattening has rendered certain optimizations, like exploiting locality of 
reference,
impossible.

\subsection{Moderate flattening}
To determine the amount of parallelism to exploit, we need a closer loke at
GPUs. \textit{compute kernel}s, simply refered to as \textit{kernel}s in this
report, are the routines that are run on parallel hardware such as GPUs.  In 
low level languages such as \texttt{OpenCl C} a kernel is a function the
programmer defines. The function is then executed \textit{N $\times$ M} times 
in parallel by
\textit{N} different \textit{workgroups} each with \textit{M}  
\textit{threads}.
The matrix-matrix multiplication in appendix \ref{cuda-matmult} is an example
of a kernel in the \texttt{CUDA} language.

A kernel can also be thought of as the work, that is to be done, in a perfect
\textit{parallel nest}. A perfect parallel nest is a nest where all the work,
is at the innermost level. Some contrived examples of perfect and imperfect
nests can be seen in Figure \ref{nests}.

\begin{figure}[h]
  \centering
  \subfloat[][A perfect \texttt{map} nest in Haskell, multiplying every element 
  of a two dimensional array by 2.]{
  \centering
  \begin{minipage}{3.5cm}
\begin{center}
\begin{lstlisting}
map (map (*2)) arr
\end{lstlisting}
\end{center}
\end{minipage}
  \label{perMapNest}}\hspace{2cm}
  \subfloat[][An imperfect \texttt{map} nest in Haskell, prepending 2 to every 
  array $x$, and multiplying every element of each $x$ by 2.]{
  \centering
  \begin{minipage}{6.5cm}
	\centering
\begin{lstlisting}
map (\x -> let y = 2:x in map (*2) y)
\end{lstlisting}
\end{minipage}
  \label{impMapNest}}\\
  \subfloat[][A perfect \texttt{map-scan} nest in Haskell, left-scanning every 
  element of a two dimensional array.]{
  \centering
  \begin{minipage}{3.5cm}
	\centering
	\begin{lstlisting}
map (scanl (+) 0)
\end{lstlisting}
\end{minipage}	\label{perScanNest}}\hspace{2cm}
  \subfloat[][A imperfect \texttt{map-scan} nest in Haskell, which only scans 
  an element of the array, if the first element of the nested array is over 
3.]{
  \centering
  \begin{minipage}{6.5cm}
	\centering
	\begin{lstlisting}
map (\x -> if (head) x > 3 then scanl (+) 0 x else x)
\end{lstlisting}
\end{minipage}
  \label{impScanNest}}
  \caption{Two examples of perfect and imperfect code. In \ref{perMapNest} and 
  \ref{impMapNest} we see two \texttt{map} nests, where \ref{impMapNest} is not 
perfectly nested due to the concatenation being performed in between. In the 
two \texttt{map-scan} nests of \ref{perScanNest} and \ref{impScanNest} we see 
that one is perfectly nested, while the other has a conditional expression in 
between making it imperfectly nested.}
  \label{nests}
\end{figure}
\begin{figure}[H]
\centering
  \subfloat[][Code fragment before distribution.]
  {
  \centering
  \begin{lstlisting}
map (\xs -> let y = reduce (+) 0 xs 
		in map (+y) xs)
	xss
\end{lstlisting}	\label{beforeDist} }\hspace{2cm}
  \subfloat[][Code fragment after distribution.]
  {
  \centering
  \begin{lstlisting}
let ys = map (\xs -> reduce (+) 0 xs) xss
in map (\xs y -> map (+y) xs) xss ys
\end{lstlisting}
  \label{afterDist}
  }
  \caption{Two version of a fragment of code, showing the distribution of SOACs 
  (\textit{second-order array combinators}) \cite{futhark-moderate-blog}.}
  \label{loopDist}
\end{figure}

\noindent Futhark transforms nested data-parallelism into flat structures from
which kernels can later be extracted. An example of this transformation can
be seen in figure \ref{loopDist}. We can see that the outer \texttt{map}, of
figure \ref{beforeDist}, contains more parallelism in it, in the form of the
parallel constructs \texttt{reduce} and \texttt{map}. So in the case where the
outer \texttt{map} does not saturate the GPU's capacity by itself we want to
exploit the parallelism of the \texttt{map}s body. With figure \ref{beforeDist}
this is not possible, since the outer \texttt{map} would be parallelised and
the inner sequentialized to one GPU kernel. Instead we can transform it to
figure \ref{afterDist}. Here the parallel constructs in the \texttt{map}s
body are distributed out into their own \texttt{map} nests, giving two perfect
\texttt{map} nests. Each line in \ref{afterDist} can then be translated into a
GPU kernel, the result from the first being used in the second.

This transformation is dubbed \textit{moderate flattening}
\cite{futhark-nested-para} due to its conceptual resemblance to the flattening
algorithm put forth by \citeauthor{flat} \cite{flat}. However it is moderate in
its approach only flattening the parallel construct as long as the parallelism
does not oversaturate the hardware, otherwise it efficiently sequentialize the remaining
parallelism. As we saw in Figure \ref{loopDist} Futhark flattens nested
parallel construct by extracting kernels based on certain rewrite/flattening
rules\cite{futhark-nested-para}. At the time of the implementation of moderate
flattening the algorithm that applied these rules was based on heuristics about
the structure of \texttt{map} nest contexts. These where decent approximations
but because of the nature of GPU's (having different capacities for
parallelism) there is no one size fits all.
 
It is important to note that Futhark does not at the time of this project 
support irregular data structures. This means that we cannot have a 
multidimensional array where the elements have different shapes. The reason for 
this limitation is that allowing irregular data structures introduces a time, 
and memory, overhead that can be hard for the programmer to understand or 
reason about. Making the programmer to flatten their datastructure by hand 
forces them to consider, and understand, this overhead.\cite{expansion} 

\subsection{Incremental flattening}
While moderate flattening was a good start it does not differentiate between different
hardware capabilities and size of the data being worked on. This is where 
incremental flattening comes in. In short incremental flattening statically 
generates multiple different, semantically equivalent, piecewise code versions 
of the same program based on the rules of moderate flattening (along with some 
additional ones) \cite{inc-flat}. It's then dynamically decided at run-time 
whether to further exploit the parallelism or not.
\begin{figure}[h]
\centering
\lstset{language=haskell}
\begin{lstlisting}
let dotprod [n] (xs: [n]f32) (ys: [n]f32): f32 =
reduce (+) 0f32 (map2 (*) xs ys)

let main [n][m][p] (xss: [n][m]f32) (yss: [m][p]f32): [n][p]f32 =
map (\xs -> map (dotprod xs) (transpose yss)) xss
\end{lstlisting}%
\caption{Matrix-matrix multiplication in Futhark \cite{ppopp}}
\label{matmultFuthark}
\end{figure}
To get an intuition for incremental flattening lets go back to the example of
matrix-matrix multiplication, which can be seen in Figure \ref{matmultFuthark}.
As matrix-matrix multiplication contains a lot of \texttt{map}s (as is common 
in GPU programs do) we will briefly explain the code versions generated by the 
flattening rule when a \texttt{map} containing additional parallelism, is 
encountered (CV is short for code version).
\begin{itemize}
\item[CV0] The body of the map will be executed sequentially. This means that 
there will be assigned one GPU thread to each element of the array being mapped 
over, each thread executing the map body sequentially.
\item[CV1] The body of the map is partially executed in parallel. This means 
that there will be assigned one GPU workgroup to each element of the 
array being mapped over. This is partial since it does not exploit the 
parallelism fully as there will most likely not be enough threads to fully 
parallelise the entire construct within the \texttt{map} body.
\item[$\cdots$] Continues to flatten the \texttt{map} function, since we still 
have further parallel capacity to exploit.  \end{itemize}
\begin{center}
  \centering \begin{forest}
[\texttt{map}
	[CV0,edge]
	[additional parallelism,edge
		[CV1,edge]
		[CV2,edge]	
	]
]
\end{forest}
  \captionof{figure}{A tree showing the structure of the three different code 
  versions generate by flattening a \texttt{map} nest}
  \label{maptree}
\end{center}
In matrix-matrix multiplication, there are three levels of nested parallelism 
to exploit: the outer \texttt{map}, the inner \texttt{map} and the 
\texttt{dotprod} function. Lets look at the different code versions Futhark 
generates, each code version denoted by a V \cite{inc-flat}.
\begin{itemize}
  \item[V0)] The outer \texttt{map} is distributed out across the parallel 
  construct executing its body sequentially. More specifically one thread 
  calculates one row in the resulting matrix each. This is CV0 for 
  \texttt{map}.
  \item[V1)] The outer \texttt{map} is executed in parallel, and the inner 
  \texttt{map} of \texttt{main} is executed partially in parallel. More 
specifically one workgroup calculates one row in the resulting matrix. This is 
CV1 for \texttt{map}.
  \item[V2)] The two outer maps of \texttt{main} are executed in parallel and 
their body (\texttt{dotprod}) is executed sequentially. More specifically each 
thread is calculating one element in the resulting matrix. This is CV0 for 
\texttt{map}.
  \item[V3)] The two outer \texttt{maps}, does not saturate the GPU, but the 
  \texttt{dotprod} function would oversaturate it, therefore, \texttt{dotprod} 
  is partially executed in parallel. More specifically each element of the 
  result matrix is executed by a workgroup. This is CV1
  \item[V4)] If the entire parallel nest does not oversaturate the GPU, the 
entire nest is executed in parallel. This means there is a thread for each 
multiplication in the \texttt{dotprod} functions.
\end{itemize}

The optimal choice for  which of these semantically equivalent versions gets 
executed depends on hardware and the input data. So each version is guarded by 
a choice that depends on these factors. To visualize the choices, we will look 
at them as a tree. In Figure \ref{MatMultTree}, we see that the matrix 
multiplication program has four choices that  are dependent on each 
other.
\begin{center}
  \centering \begin{forest}
[Choice0
	[V0,edge ]
	[Choice1,edge 
		[V1,edge ]
		[Choice2,edge 
			[V2,edge ]
			[Choice3,edge 
				[V3,edge ]
				[V4,edge ]
			]
		]	
	]
]
\end{forest}
  \captionof{figure}{The structure of choices found in the futhark program for 
  matrix-matrix multiplication (see listing \ref{matmultFuthark}). (V) 
represents the resulting code version of a choice}
  \label{MatMultTree}
\end{center}
With these dependent choices, the idea of incremental flattening becomes 
increasingly clear. We step through the code, and decided whether we should 
flatten the parallelism available, or whether we should sequentialize it and 
exploit locality-of-reference optimizations. Due to the dependency of the 
choices, we end up iterating through the parallel nest, and flattening until we 
reach the full capacity of the hardware, hence the name \textit{incremental 
flattening}. The choices will not always be in form of a unbalanced tree 
(although they are for the most part), they can also be balanced, and/or form a 
forest, we will give an example of a more complicated tree later.

\subsection{Structure of the code versions}
%Bad section name
To further understand how we will choose the correct code version, we have 
filled out the matrix-matrix multiplication tree, with more detail, which can 
be seen in Figure \ref{MatMultTreeFilled}. In the tree the non-leaf nodes are 
the statically generated predicates (the choice), hence forth known as 
\textit{threshold comparisons} or just \textit{comparisons}. the comparisons  
guard the different code versions, ensuring the sematics of the program does 
not change. The comparisons have a \textit{threshold parameter} on the left 
$t_i$, which is a symbolic representation of the parallel capacity of the 
hardware. On the right we have a value that reflects the shape and size of the 
data, as seen in the figure.

\begin{figure}
  \centering \begin{forest}
[$t_0\leq dVal_0$
	[V0,edge label={node[midway,left,font=\scriptsize]{T}}]
	[$t_1\leq dVal_1$,edge label={node[midway,right,font=\scriptsize]{F}}
		[V1,edge label={node[midway,left,font=\scriptsize]{T}}]
		[$t_2\leq dVal_2$,edge label={node[midway,right,font=\scriptsize]{F}}
			[V2,edge label={node[midway,left,font=\scriptsize]{T}}]
			[$t_3\leq dVal_3$,edge label={node[midway,right,font=\scriptsize]{F}}
				[V3,edge label={node[midway,left,font=\scriptsize]{T}}]
				[V4,edge label={node[midway,right,font=\scriptsize]{F}}]
			]
		]	
	]
]
\end{forest}
  \caption{The tree generated by matrix-matrix multiplication.  ($t_i$) is a 
  threshold, ($n$, $p$, and $m$) are the sizes of the matricies show in the 
code (see \ref{matmultFuthark}), and (T) and (F) are indicative of the path we 
take, based on the threshold comparison.}
  \label{MatMultTreeFilled}
\end{figure}

To inspect the structure of these predicates and thresholds parameters further, 
lets look at a more complex Futhark program, from the Futhark benchmarks, 
called \textit{LocVolCalib}.
\begin{figure}
  \centering
  \begin{forest}
[Th4
	[V1,edge label={node[midway,left,font=\scriptsize]{T}}]
	[Th5,edge label={node[midway,right,font=\scriptsize]{F}}
		[V2,edge label={node[midway,left,font=\scriptsize]{T}}]
		[Th6,edge label={node[midway,right,font=\scriptsize]{F}}
			[V3,edge label={node[midway,left,font=\scriptsize]{T}}]
			[Th7,edge label={node[midway,right,font=\scriptsize]{F}}
				[Th16,edge label={node[midway,left,font=\scriptsize]{F}}
					[V5,edge label={node[midway,left,font=\scriptsize]{T}}]
					[Th17,edge label={node[midway,right,font=\scriptsize]{F}}
						[V6,edge label={node[midway,left,font=\scriptsize]{T}}]
						[V7,edge label={node[midway,right,font=\scriptsize]{F}}]
					]
				]
				[Th8,edge label={node[midway,left,font=\scriptsize]{F}}
					[V8,edge label={node[midway,left,font=\scriptsize]{T}}]
					[Th9,edge label={node[midway,right,font=\scriptsize]{F}}
						[V9,edge label={node[midway,left,font=\scriptsize]{T}}]
						[V10,edge label={node[midway,right,font=\scriptsize]{F}}]
					]
				]
			[V4,edge label={node[midway, right,font=\scriptsize]{T}}]
			]
		]
	]	
]
\end{forest}
  \caption{The dependencies between thresholds, of the test program 
  \texttt{LocVolCalib.fut}. (Th) is a threshold comparison, (V) is a code 
version, and (T) and (F) are indicative of the path we take, based on the 
threshold comparison}
  \label{LocVolCalibTree}
\end{figure}
\noindent It is important not to get an end node (code version) confused with a 
complete program. While this can be the case, we will give two examples of 
paths, and corresponding code versions, through the tree in Figure 
\ref{LocVolCalibTree}, to illustrate this;
\begin{itemize}
  \item \texttt{\{(Th4, False), (Th5, False), (Th6, False), (Th7, True)\} $\to$ 
    V4}
  \item \texttt{\{(Th4, False), (Th5, False), (Th6, False), (Th7, False), (Th8, 
    False), (Th16, False), (Th9, True), (Th17, True)\} $\to$ \{V6, V9\}}
\end{itemize} 

The first path is simple and ends in a single code version, \texttt{V4}, that 
is semantically equivalent to the original program. The second path is more 
interesting, \texttt{T7} has two child nodes, that are reached when the 
comparison is false. Here it is clear that two end nodes are reached, namely 
\texttt{(V6, V9)}, and these two code versions are both executed, and together they 
are semantically equivalent to the original program.  This is also important to 
note, because we could have a forest, instead of a single tree and this would 
lead to multiple code versions that are all executed, together being semantically 
equivalent to the original Futhark program. 
