\forestset{sn edges/.style={for tree={parent anchor=south, child anchor=north}}}

\begin{figure}[h]
\centering
	\begin{minipage}{0.25\textwidth}
	\centering
	\begin{forest}
	[\texttt{(NULL, false)}]
	\end{forest}
	\end{minipage}\hspace*{\fill}
	\begin{minipage}{0.25\textwidth}
	\centering
	\begin{forest}
	[\texttt{(NULL, false)}
		[\texttt{(Th4, false)}]
	]
	\end{forest}
	\end{minipage}\hspace*{\fill}
	\begin{minipage}{0.25\textwidth}
	\centering
	\begin{forest}
	[\texttt{(NULL, false)}
		[\texttt{(Th4, false)}
			[\texttt{(Th5, false)}
			]
		]
	]
	\end{forest}
	\end{minipage}\hspace*{\fill}
	\begin{minipage}{0.25\textwidth}
	\centering
	\begin{forest}
	[\texttt{(NULL, false)}
		[\texttt{(Th4, false)}
			[\texttt{(Th5, false)}
				[\texttt{(Th6, false)}]
			]
		]
	]
	\end{forest}
	\end{minipage}
	\captionof{figure}{The recusive steps of building the tree structure for the thresholds, with (Th) being a threshold parameter. The list of thresholds for the tree would be \texttt{[main.suff\_intra\_par\_5 (threshold (!main.suff\_outer\_par\_4)), main.suff\_outer\_par\_4 (threshold ()), main.suff\_outer\_par\_6 (threshold (!main.suff\_outer\_par\_4 !main.suff\_intra\_par\_5))]}}
	%We might want to remove the list of thresholds
	\label{buildThresTree}
\end{figure}
