\forestset{
  colour my roots/.style={
    before typesetting nodes={
      edge=#1,
      for ancestors={
        edge=#1,
        #1,
      },
      #1,
    }
  },
  my edge label/.style={
    edge label={
      node [midway, fill=white, font=\scriptsize] {#1}
    }
  }
}

\begin{figure}[h]
\centering
	\subfloat[][small]{
	\centering
	\begin{forest}
[Th4
	[V1,edge label={node[midway,left,font=\scriptsize]{T}}]
	[Th5,edge label={node[midway,right,font=\scriptsize]{F}}
		[V2,edge label={node[midway,left,font=\scriptsize]{T}}]
		[Th6,edge label={node[midway,right,font=\scriptsize]{F}}
			[V3,colour my roots=red, edge label={node[midway,left,font=\scriptsize]{T}}]
			[Th7,edge label={node[midway,right,font=\scriptsize]{F}}
				[Th16,edge label={node[midway,left,font=\scriptsize]{F}}
					[V5,edge label={node[midway,left,font=\scriptsize]{T}}]
					[Th17,edge label={node[midway,right,font=\scriptsize]{F}}
						[V6, edge label={node[midway,left,font=\scriptsize]{T}}]
						[V7,edge label={node[midway,right,font=\scriptsize]{F}}]
					]
				]
				[Th8,edge label={node[midway,left,font=\scriptsize]{F}}
					[V8,edge label={node[midway,left,font=\scriptsize]{T}}]
					[Th9,edge label={node[midway,right,font=\scriptsize]{F}}
						[V9, edge label={node[midway,left,font=\scriptsize]{T}}]
						[V10,edge label={node[midway,right,font=\scriptsize]{F}}]
					]
				]
			[V4, edge label={node[midway, right,font=\scriptsize]{T}}]
			]
		]
	]	
]
\end{forest}
	\label{combinationA}}
	\subfloat[][medium]{
	\centering
		\begin{forest}
[Th4
	[V1,edge label={node[midway,left,font=\scriptsize]{T}}]
	[Th5,edge label={node[midway,right,font=\scriptsize]{F}}
		[V2,edge label={node[midway,left,font=\scriptsize]{T}}]
		[Th6,edge label={node[midway,right,font=\scriptsize]{F}}
			[V3,edge label={node[midway,left,font=\scriptsize]{T}}]
			[Th7,edge label={node[midway,right,font=\scriptsize]{F}}
				[Th16,edge label={node[midway,left,font=\scriptsize]{F}}
					[V5,edge label={node[midway,left,font=\scriptsize]{T}}]
					[Th17,edge label={node[midway,right,font=\scriptsize]{F}}
						[V6, edge label={node[midway,left,font=\scriptsize]{T}}]
						[V7,edge label={node[midway,right,font=\scriptsize]{F}}]
					]
				]
				[Th8,edge label={node[midway,left,font=\scriptsize]{F}}
					[V8,edge label={node[midway,left,font=\scriptsize]{T}}]
					[Th9,edge label={node[midway,right,font=\scriptsize]{F}}
						[V9, edge label={node[midway,left,font=\scriptsize]{T}}]
						[V10,edge label={node[midway,right,font=\scriptsize]{F}}]
					]
				]
			[V4,colour my roots=green,edge label={node[midway, right,font=\scriptsize]{T}}]
			]
		]
	]	
]
\end{forest}
	\label{combinationB}}
	\subfloat[][large]{
	\centering
	\begin{forest}
[Th4
	[V1,edge label={node[midway,left,font=\scriptsize]{T}}]
	[Th5,edge label={node[midway,right,font=\scriptsize]{F}}
		[V2,edge label={node[midway,left,font=\scriptsize]{T}}]
		[Th6,edge label={node[midway,right,font=\scriptsize]{F}}
			[V3,edge label={node[midway,left,font=\scriptsize]{T}}]
			[Th7,edge label={node[midway,right,font=\scriptsize]{F}}
				[Th16,edge label={node[midway,left,font=\scriptsize]{F}}
					[V5,edge label={node[midway,left,font=\scriptsize]{T}}]
					[Th17, edge label={node[midway,right,font=\scriptsize]{F}}
						[V6, colour my roots=blue, edge label={node[midway,left,font=\scriptsize]{T}}]
						[V7, edge label={node[midway,right,font=\scriptsize]{F}}]
					]
				]
				[Th8,edge label={node[midway,left,font=\scriptsize]{F}}
					[V8,edge label={node[midway,left,font=\scriptsize]{T}}]
					[Th9,edge label={node[midway,right,font=\scriptsize]{F}}
						[V9 ,colour my roots=blue, edge label={node[midway,left,font=\scriptsize]{T}}]
						[V10, edge label={node[midway,right,font=\scriptsize]{F}}]
					]
				]
			[V4,edge label={node[midway, right,font=\scriptsize]{T}}]
			]
		]
	]	
]
\end{forest}
	\label{combinationC}}
	\caption{We see an example of how three dataset might choose different paths
    through the tree depending on their size, with the same set of threshold
    parameters. The general idea is that the
    longer down we go in the tree the more parallelism is utilized. The three trees here show a unique path through each, and this combination of these three paths, is also unique. We need to check all of such unique combinations. This tree is
    based on the \textit{LocVolCalb} program. The paths show here are not necessarily the
    combination of paths any dataset would take based on any combination of
    threshold values.}
	\label{combination}
\end{figure}
