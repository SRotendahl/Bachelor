\section*{Abstract}
This report describes the implementation of an autotuner for the programming language Futhark. This particular autotuner is implemented to test the notion of Futhark programs being small enough to be comfortably autotuned, by exhaustively searching for the best execution. 

The autotuner is shown to work, always picking the best execution, achieving executions up to 26.97 times faster, while in some cases giving the same result compared to not tuning, as well as slightly outperforming Futharks existing autotuner. 

However the exhaustive autotuner is slow at autotuning. We see that the amount of datasets and the variance in them, is crucial for the amount of time spend autotuning. For large programs with multiple dataset it took 23 hours to autotune. While we present an optimization for the implementation, and present a way to possibly midigate the variance and amount of datasets, we conclude that, currently, exhaustively autotuning Futhark is not comfortable due to the long time required to do so.