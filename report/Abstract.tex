\section*{Abstract}
This report describes the implementation of an autotuner for the programming
language Futhark. This particular autotuner is implemented to test the notion
of Futhark programs being small enough to be comfortably autotuned by
exhaustively searching all executions for the best execution. 

The autotuner is shown to work, always picking the best execution, achieving
executions up to 26.97 times faster, while in some cases giving the same result
compared to not tuning, as well as slightly outperforming Futharks existing
autotuner.

However the exhaustive autotuner is slow at autotuning. We see that the amount
of datasets and the variance in them, is crucial for the amount of time spend
autotuning. The autotuner performed very well if there was only a single
dataset, but for large programs with multiple dataset it took 23 hours to
autotune. 

We conclude that an exhaustive autotuner is not suitable, if the underlying concept for tuning
is to add a multitude of fairly random datasets and tune the parameters from that. 
However if it turns out to be possible to create more representative datasets,
that can reduce the number of datasets needed to autotune a program
then an exhaustive autotuner might be feasible. 

