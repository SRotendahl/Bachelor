\documentclass[11pt,a4paper,titlepage]{article}

\usepackage[utf8]{inputenc}
\usepackage[english]{babel}
%\usepackage{geometry}
\usepackage[margin=2cm,headheight=13.6cm]{geometry}
\usepackage{color}
\usepackage{listings}
\usepackage{mdwlist}
\usepackage{graphicx}
\usepackage{xcolor}
\usepackage{geometry}
\usepackage[backend=biber]{biblatex}
\usepackage{textcomp}
\usepackage{paralist}
\usepackage{caption}
\usepackage{tcolorbox}
\usepackage{amsmath}
\usepackage{verbatim}
%\usepackage{float}
\usepackage{titling}
\usepackage{pgfkeys}
\usepackage{enumitem}
\usepackage{forest}
%\usepackage{subfig}
\usepackage{hyperref}
\usepackage[final]{pdfpages}
\usepackage{multicol}

\definecolor{OliveGreen}{cmyk}{0.64,0,0.95,0.40}
\definecolor{CadetBlue}{cmyk}{0.62,0.57,0.23,0}
\definecolor{lightlightgray}{gray}{0.9}
 
 
 
\lstset{
language=java,                          % Code langugage
basicstyle=\ttfamily,                   % Code font, Examples: \footnotesize, \ttfamily
keywordstyle=\color{OliveGreen},        % Keywords font ('*' = uppercase)
commentstyle=\color{gray},              % Comments font
numbers=left,                           % Line nums position
numberstyle=\tiny,                      % Line-numbers fonts
stepnumber=1,                           % Step between two line-numbers
numbersep=5pt,                          % How far are line-numbers from code
backgroundcolor=\color{lightlightgray}, % Choose background color
frame=none,                             % A frame around the code
tabsize=2,                              % Default tab size
captionpos=t,                           % Caption-position = bottom
breaklines=true,                        % Automatic line breaking?
breakatwhitespace=false,                % Automatic breaks only at whitespace?
showspaces=false,                       % Dont make spaces visible
showtabs=false,                         % Dont make tabls visible
columns=flexible,                       % Column format
morekeywords={__global__, __device__, __syncthreads},  % CUDA specific keywords
}
%numbers=left,xleftmargin=2em,frame=single,framexleftmargin=1.5em
\lstdefinestyle{haskell}
{
  %frame=none,
  stepnumber=1,
  numbers=left,
  numbersep=5pt,
  numberstyle=\ttfamily\tiny\color[gray]{0.3},
  belowcaptionskip=\bigskipamount,
  captionpos=b,
  escapeinside={*'}{'*},
  language=haskell,
  tabsize=2,
  emphstyle={\bf},
  commentstyle=\it,
  stringstyle=\mdseries\rmfamily,
  showspaces=false,
  keywordstyle=\bfseries\rmfamily,
  columns=flexible,
  basicstyle=\small\sffamily,
  showstringspaces=false,
  morecomment=[l]\%,
  xleftmargin=10.0ex
}


\definecolor{mygreen}{RGB}{0,127,0}
\definecolor{mygray}{RGB}{100,100,100}
\definecolor{mymauve}{RGB}{100,32,255}
\definecolor{lgray}{RGB}{230,230,230}
\lstset{ %
  frame=none,
  backgroundcolor=\color{white},   % choose the background color; you must add \usepackage{color} or \usepackage{xcolor}
  basicstyle=\footnotesize\ttfamily,        % the size of the fonts that are used for the code
  breakatwhitespace=false,         % sets if automatic breaks should only happen at whitespace
  breaklines=true,                 % sets automatic line breaking
  captionpos=t,                    % sets the caption-position to bottom
  commentstyle=\color{mygreen},    % comment style
  deletekeywords={...},            % if you want to delete keywords from the given language
  escapeinside={\%*}{*)},          % if you want to add LaTeX within your code
  extendedchars=true,              % lets you use non-ASCII characters; for 8-bits encodings only, does not work with UTF-8
  keepspaces=true,                 % keeps spaces in text, useful for keeping indentation of code (possibly needs columns=flexible)
  keywordstyle=\color{blue},       % keyword style
  language=,                 % the language of the code
  morekeywords={*,...},            % if you want to add more keywords to the set
  numbers=left,                    % where to put the line-numbers; possible values are (none, left, right)
  numbersep=5pt,                   % how far the line-numbers are from the code
  numberstyle=\tiny\color{mygray}, % the style that is used for the line-numbers
  rulecolor=\color{black},         % if not set, the frame-color may be changed on line-breaks within not-black text (e.g. comments (green here))
  showspaces=false,                % show spaces everywhere adding particular underscores; it overrides 'showstringspaces'
  showstringspaces=false,          % underline spaces within strings only
  showtabs=false,                  % show tabs within strings adding particular underscores
  stepnumber=1,                    % the step between two line-numbers. If it's 1, each line will be numbered
  stringstyle=\color{mymauve},     % string literal style
  tabsize=4,                       % sets default tabsize to 2 spaces
  aboveskip=3mm,
  belowskip=3mm,
}

\usepackage{fancyhdr}
\pagestyle{fancy}
\rhead{Simon Rotendahl \& Carl Mathias Graae Larsen}
\usepackage{datetime}
\usepackage{moresize}

\newdateformat{monthyeardate}{%
  \monthname[\THEMONTH] \THEYEAR}

\newcommand{\rulebreak}{%
	\par%
	\vspace{0.9cm}%
    \noindent\rule{4cm}{0.4pt}%
    \vspace{1.2cm}%
    \par%
}

\newcommand{\coverpage}[1]{%
	\pagenumbering{roman}%
	\thispagestyle{empty}%
	\lhead{\textsc{Bachelors project}}%
    \title{Auto-tuning Futhark}%
    \author{Simon Rotendahl (mpx651) \& Carl Mathias Graae Larsen (pwh334)\\ \{\textit{simon, cala}\}@di.ku.dk}%
    \newgeometry{left=2cm,bottom=0cm,right=2cm,top=0cm}%
	\begin{center}\hspace{0pt}\vfill%
    \uppercase{Department of Computer Science at The University of Copenhagen}
	\rulebreak%
    {\Large\textbf{Bachelors project}}
    
    \vspace{0.5cm}
    {\HUGE\textbf{\textit{#1}}}
    
    \vspace{0.5cm}
	\theauthor%
	\par%
	\vspace{0.9cm}%
    \noindent\rule{4cm}{0.4pt}%
    \vspace{0.45cm}
    % If TOC gets to long, uncomment multicols for columns
    %\begin{multicols}{2}
	\tableofcontents
	%\end{multicols}
	\rulebreak%
    \monthyeardate\today\par
    \hspace{0pt}
	\end{center}%
    \vfill
    \hspace{0pt}
	\pagebreak%
    \restoregeometry%
    \pagenumbering{arabic}%
}

\addbibresource{References.bib}
