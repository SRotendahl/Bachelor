% Opsætter KU Tex
%%%%%%%%%%%%%%%%%%%%%%%%%%%%%%%%%%%%%%%%%%%%%%%%%%%%%%%%%%%%%%%%%%%%%%%%%%%%%%%%
\documentclass{article}
\usepackage[a4paper, hmargin={2.8cm, 2.8cm}, vmargin={2.5cm, 2.5cm}]{geometry}
\usepackage{eso-pic}  % \AddToShipoutPicture
\usepackage{graphicx} % \includegraphics
\usepackage{url}
%%%%%%%%%%%%%%%%%%%%%%%%%%%%%%%%%%%%%%%%%%%%%%%%%%%%%%%%%%%%%%%%%%%%%%%%%%%%%%%%

\usepackage[utf8]{inputenc}
\usepackage[english]{babel}
\usepackage{geometry}
\usepackage{color}
\usepackage{listings}
\usepackage{mdwlist}
\usepackage{xcolor}
\usepackage[backend=biber]{biblatex}
\usepackage{textcomp}
\usepackage{paralist}
\usepackage{tcolorbox}
\usepackage{verbatim}
\usepackage{float}
\usepackage{titling}
\usepackage{pgfkeys}
\usepackage{enumitem}
\usepackage{forest}
\usepackage{subfig}
\usepackage{hyperref}
\usepackage[final]{pdfpages}
\usepackage{multicol}

\definecolor{OliveGreen}{cmyk}{0.64,0,0.95,0.40}
\definecolor{CadetBlue}{cmyk}{0.62,0.57,0.23,0}
\definecolor{lightlightgray}{gray}{0.9}
 
 
 
\lstset{
language=java,                          % Code langugage
basicstyle=\ttfamily,                   % Code font, Examples: \footnotesize, \ttfamily
keywordstyle=\color{OliveGreen},        % Keywords font ('*' = uppercase)
commentstyle=\color{gray},              % Comments font
numbers=left,                           % Line nums position
numberstyle=\tiny,                      % Line-numbers fonts
stepnumber=1,                           % Step between two line-numbers
numbersep=5pt,                          % How far are line-numbers from code
backgroundcolor=\color{lightlightgray}, % Choose background color
frame=none,                             % A frame around the code
tabsize=2,                              % Default tab size
captionpos=t,                           % Caption-position = bottom
breaklines=true,                        % Automatic line breaking?
breakatwhitespace=false,                % Automatic breaks only at whitespace?
showspaces=false,                       % Dont make spaces visible
showtabs=false,                         % Dont make tabls visible
columns=flexible,                       % Column format
morekeywords={__global__, __device__, __syncthreads},  % CUDA specific keywords
}
%numbers=left,xleftmargin=2em,frame=single,framexleftmargin=1.5em
\lstdefinestyle{haskell}
{
  %frame=none,
  stepnumber=1,
  numbers=left,
  numbersep=5pt,
  numberstyle=\ttfamily\tiny\color[gray]{0.3},
  belowcaptionskip=\bigskipamount,
  captionpos=b,
  escapeinside={*'}{'*},
  language=haskell,
  tabsize=2,
  emphstyle={\bf},
  commentstyle=\it,
  stringstyle=\mdseries\rmfamily,
  showspaces=false,
  keywordstyle=\bfseries\rmfamily,
  columns=flexible,
  basicstyle=\small\sffamily,
  showstringspaces=false,
  morecomment=[l]\%,
  xleftmargin=10.0ex
}


\definecolor{mygreen}{RGB}{0,127,0}
\definecolor{mygray}{RGB}{100,100,100}
\definecolor{mymauve}{RGB}{100,32,255}
\definecolor{lgray}{RGB}{230,230,230}
\lstset{ %
  frame=none,
  backgroundcolor=\color{white},   % choose the background color; you must add \usepackage{color} or \usepackage{xcolor}
  basicstyle=\footnotesize\ttfamily,        % the size of the fonts that are used for the code
  breakatwhitespace=false,         % sets if automatic breaks should only happen at whitespace
  breaklines=true,                 % sets automatic line breaking
  captionpos=t,                    % sets the caption-position to bottom
  commentstyle=\color{mygreen},    % comment style
  deletekeywords={...},            % if you want to delete keywords from the given language
  escapeinside={\%*}{*)},          % if you want to add LaTeX within your code
  extendedchars=true,              % lets you use non-ASCII characters; for 8-bits encodings only, does not work with UTF-8
  keepspaces=true,                 % keeps spaces in text, useful for keeping indentation of code (possibly needs columns=flexible)
  keywordstyle=\color{blue},       % keyword style
  language=,                 % the language of the code
  morekeywords={*,...},            % if you want to add more keywords to the set
  numbers=left,                    % where to put the line-numbers; possible values are (none, left, right)
  numbersep=5pt,                   % how far the line-numbers are from the code
  numberstyle=\tiny\color{mygray}, % the style that is used for the line-numbers
  rulecolor=\color{black},         % if not set, the frame-color may be changed on line-breaks within not-black text (e.g. comments (green here))
  showspaces=false,                % show spaces everywhere adding particular underscores; it overrides 'showstringspaces'
  showstringspaces=false,          % underline spaces within strings only
  showtabs=false,                  % show tabs within strings adding particular underscores
  stepnumber=1,                    % the step between two line-numbers. If it's 1, each line will be numbered
  stringstyle=\color{mymauve},     % string literal style
  tabsize=4,                       % sets default tabsize to 2 spaces
  aboveskip=3mm,
  belowskip=3mm,
}



% Pakker til skrifttyper, tekst osv.
%%%%%%%%%%%%%%%%%%%%%%%%%%%%%%%%%%%%%%%%%%%%%%%%%%%%%%%%%%%%%%%%%%%%%%%%%%%%%%%%
\usepackage[utf8]{inputenc} % Implementere Unicode
\usepackage[T1]{fontenc}    % Unicode skrifttype, fx. é skrives som 1 tegn
\usepackage{microtype}      % Forbedre linjeombrydningen
\usepackage{libertine}      % Skrifttype
\usepackage[scaled=0.83]{inconsolata} % Skrifttype til kode til kode
%%%%%%%%%%%%%%%%%%%%%%%%%%%%%%%%%%%%%%%%%%%%%%%%%%%%%%%%%%%%%%%%%%%%%%%%%%%%%%%%

% Pakker til matematik og kode.
%%%%%%%%%%%%%%%%%%%%%%%%%%%%%%%%%%%%%%%%%%%%%%%%%%%%%%%%%%%%%%%%%%%%%%%%%%%%%%%%
\usepackage{mathtools}       % Udvidelse til amsmath pakken
\usepackage{algpseudocode}   % pseudocode til algoritmer
\usepackage{algorithm}       % Pakke til algoritmer
\usepackage{amsthm}          % Pakke til Theroms
\usepackage{algorithmicx}
\usepackage{amssymb}
%%%%%%%%%%%%%%%%%%%%%%%%%%%%%%%%%%%%%%%%%%%%%%%%%%%%%%%%%%%%%%%%%%%%%%%%%%%%%%%%
%\usepackage{subcaption}
\usepackage{subfig}
% Pakker til layout.
%%%%%%%%%%%%%%%%%%%%%%%%%%%%%%%%%%%%%%%%%%%%%%%%%%%%%%%%%%%%%%%%%%%%%%%%%%%%%%%%
\usepackage{fancyhdr}        % Gør det muligt at bruge sidehoveder
\usepackage{graphicx}        % Mulighed for bl.a. \includegraphics
\usepackage{parskip}         % Første paragraf i afsnit indrykkes ikke
\usepackage{listings}        % Pakke til at indsætte kode
\usepackage{enumitem}        % Gør det muligt at tilpasse lister
\usepackage{titlesec}        % Tilpassing af afstand mellem sektioner
\usepackage[lastpage,user]{zref} % Side x af y
\usepackage{caption}
\usepackage{scrextend}
%%%%%%%%%%%%%%%%%%%%%%%%%%%%%%%%%%%%%%%%%%%%%%%%%%%%%%%%%%%%%%%%%%%%%%%%%%%%%%%%

% Implementerer en række makroer og de pakker der er importeret
%%%%%%%%%%%%%%%%%%%%%%%%%%%%%%%%%%%%%%%%%%%%%%%%%%%%%%%%%%%%%%%%%%%%%%%%%%%%%%%%
\pagestyle{fancy}                        % Implementerer sidehoved
\lhead{BSc Thesis} % Venstre sidehoved
\rhead{Simon Rotendahl \& Carl Mathias Graae Larsen}  % Højre sidehoved
\cfoot{\thepage\ of \zpageref{LastPage}} % Side x af y
\newtheorem*{prp}{Propostion}            % Skaber nyt theorem
\setlist{nolistsep}              % Formindsker mellemrum mellem listepunkter

% Definitioner af farver
%%%%%%%%%%%%%%%%%%%%%%%%%%%%%%%%%%%%%%%%%%%%%%%%%%%%%%%%%%%%%%%%%%%%%%%%%%%%
\definecolor{KURed1}{RGB}{144,26,30}    % Official KU Red 1
\definecolor{KURed2}{RGB}{199,36,41}    % Unofficial KU Red
\definecolor{KUGray1}{RGB}{102,102,102} % Official KU Gray 1
\definecolor{KUGray2}{RGB}{133,133,133} % Official KU Gray 2
\definecolor{KUGray3}{RGB}{163,163,163} % Official KU Gray 3
\definecolor{KUGray4}{RGB}{194,194,194} % Official KU Gray 4
\definecolor{KUGray5}{RGB}{224,224,224} % Official KU Gray 5
%%%%%%%%%%%%%%%%%%%%%%%%%%%%%%%%%%%%%%%%%%%%%%%%%%%%%%%%%%%%%%%%%%%%%%%%%%%%

% Laver titel
%%%%%%%%%%%%%%%%%%%%%%%%%%%%%%%%%%%%%%%%%%%%%%%%%%%%%%%%%%%%%%%%%%%%%%%%%%%%

\addbibresource{References.bib}

\title{
  \vspace{13em}
  \large{BSc Thesis} \\
  \Huge{Autotuning Futhark} \\
}

\author{
  Simon Rotendahl --- \textit{mpx651@ku.dk} \\ Carl Mathias Graae Larsen --- \textit{pwh334@ku.dk}
}

\date{
  \vspace{22em}
  \today \\
Supervisor: Troels Henriksen
}

