\section{Design \& Implementation}
\label{imp}
Our goal when autotuning a Futhark program is to find the set of values for the
threshold parameters that will minimize the running time of the program on the
given datasets. There is no way to be completely sure
which combination of values will lead to the fastest running time on the given
datasets, except to try them all out. This is called an exhaustive search, and
it is the way we will find the optimal set of values for the threshold
parameters. 

As we saw in Section \ref{BabyGotBack} the threshold parameters within a
program exhibit a tree structure of dependencies, see Figure \ref{LocVolCalibTree}.
Therefore we will start by explaining how we build a structure analogous to these.
%
\subsection{Building the Tree}
Every, compiled, Futhark program can be given a
flag\footnote{\texttt{--print-sizes}. The environment variable
\texttt{FUTHARK\_INCREMENTAL\_FLATTENING=1} needs to be set first.} so the
thresholds, and their structure, is printed out. The syntax of the string
representing each threshold parameter can be seen in Figure
\ref{thresholdSyntax}.
\begin{figure}[h]
	$$\underbrace{\texttt{main.suff\_outer\_par\_6}}_\text{threshold parameter \#6} \overbrace{\texttt{(threshold (!main.suff\_outer\_par\_4} \underbrace{\texttt{!main.suff\_intra\_par\_5))}}_\text{(!) indicates a false comparison}}^\text{list of thresholds that \#6 is dependent on}$$
	\caption{The syntax of thresholds, printed by Futhark.}
	\label{thresholdSyntax}
\end{figure}
We translate the list of these strings into a tree structure similar to the
trees we have seen in Section \ref{BabyGotBack}. 
We start by finding the thresholds that does not depend on any other thresholds, 
being true or false, in order for them to be evaluated.
%this will make them root nodes in their respective trees. 
These thresholds will be root nodes of their respective trees.
In it's current form Futhark will only generate one of these root nodes. In
order to make the autotuner adaptable to future changes to the compiler we take
into account the possibility that the structure of the thresholds will be a
forest. We solve this by creating a root node that has no useful information,
called \textit{nullroot}, that will serve as the parent of the forest, this
allows us to process a forest the same way we would a tree.

To find the children of one of the root nodes we look for all the thresholds
that depend of the root, and make them children of it in the tree. To get their
children we look for the thresholds that depend on the node itself and the root that
was it's parent. This is done for each layer until there are no more thresholds
that can be put as children of a node.

Unlike the theoretical trees, see Section \ref{BabyGotBack}, the implemented trees cannot carry information on
the edges themselves. This means the information is instead encoded in the
children. So each node is a tuple containing the name of the threshold
parameter and a boolean denoting the result of the parents threshold
comparison that lead to the node. The process of constructing the tree can be seen in Figure \ref{buildThresTree}.
\forestset{sn edges/.style={for tree={parent anchor=south, child anchor=north}}}

\begin{figure}[h]
\centering
	\begin{minipage}{0.25\textwidth}
	\centering
	\begin{forest}
	[\texttt{(NULL, false)}]
	\end{forest}
	\end{minipage}\hspace*{\fill}
	\begin{minipage}{0.25\textwidth}
	\centering
	\begin{forest}
	[\texttt{(NULL, false)}
		[\texttt{(Th4, false)}]
	]
	\end{forest}
	\end{minipage}\hspace*{\fill}
	\begin{minipage}{0.25\textwidth}
	\centering
	\begin{forest}
	[\texttt{(NULL, false)}
		[\texttt{(Th4, false)}
			[\texttt{(Th5, false)}
			]
		]
	]
	\end{forest}
	\end{minipage}\hspace*{\fill}
	\begin{minipage}{0.25\textwidth}
	\centering
	\begin{forest}
	[\texttt{(NULL, false)}
		[\texttt{(Th4, false)}
			[\texttt{(Th5, false)}
				[\texttt{(Th6, false)}]
			]
		]
	]
	\end{forest}
	\end{minipage}
	\captionof{figure}{The recusive steps of building the tree structure for the thresholds, with (Th) being a threshold parameter. The list of thresholds for the tree would be \texttt{[main.suff\_intra\_par\_5 (threshold (!main.suff\_outer\_par\_4)), main.suff\_outer\_par\_4 (threshold ()), main.suff\_outer\_par\_6 (threshold (!main.suff\_outer\_par\_4 !main.suff\_intra\_par\_5))]}}
	%We might want to remove the list of thresholds
	\label{buildThresTree}
\end{figure}


A possible change in the Futhark compiler that our implementation cannot handle
in its current form is if thresholds can depend on multiple direct parents.
This change would require a child to carry more information than a single
boolean. This problem could be solved by duplicating the problematic subtree,
thereby having a version of the subtree that came from a comparisons being
true, and one for comparisons being false.

\subsection{Limiting Search Space}
If we knew nothing about
the thresholds, or their structure, we would have to try every configuration of
numbers between 0 and whatever the highest number their implementation allows. 
With a simple program such a matrix-matrix multiplication (which has 4 thresholds, see Figure \ref{MatMultTreeFilled}), and assuming an upper limit of $2^{15}$ which is the threshold default, and therefore a conservative limit,
we would end up with $\left(2^{15}\right)^4 \approx 1.153\times10^{18}$ configurations.
This search would obviously take far longer than any of us would be alive, so
it's paramount that we  limit the search space.
\\
They way we limit the search space is by using our knowledge of the threshold
parameters and only trying values that would result in a different sets of code 
version being executed on at least one of the datasets. More specifically we will
use the knowledge that the comparisons are of the form
(\texttt{Threshold\_value <= comparison\_value}). Since the
\texttt{comparison\_value} is constant for a certain dataset there are effectively
only two different threshold values that need to be checked for one threshold 
in isolation. One where it's equal to the \texttt{comparison\_value} and one
where it's $\texttt{comparison\_value} + 1$, as these will result in true and
false respectively.
Of course the thresholds are not in isolation but structured as we've seen
Figure \ref{LocVolCalibTree}. So some comparison result in earlier 
thresholds will make later thresholds in the tree redundant, as they are never
compared. We can see this idea in Figure \ref{redundant}. A concrete example
would be Figure \ref{MatMultTreeFilled}, with the thresholds $t_i = [10, 20,
30, 40]$, and the matrix sizes $n=20, p=30, n\times p=600, m=50$. So the
predicate of $t_0 \leq n$ would result in taking the \texttt{true} branch,
however this is also the case for $t_0 = 11$, making these two different
configurations have the same dynamic behavior.
\forestset{sn edges/.style={for tree={parent anchor=south, child anchor=north}}}

\begin{figure}[h]
\centering
	\subfloat[][]{
	\centering
	\begin{forest}
	[$\quad$, circle, draw, edge label={node[midway,right,font=\scriptsize]{F}}
		[$\quad$, circle, draw, edge label={node[midway,right,font=\scriptsize]{F}}
			[$\quad$, circle, draw, edge label={node[midway,right,font=\scriptsize]{F}}
			]
		]
	]
	\end{forest}
	\label{redundantA}}
	\hspace*{0.25\textwidth}
	\subfloat[][]{
	\centering
	\begin{forest}
	[T, circle, draw, edge label={node[midway,right,font=\scriptsize]{F}}
		[X, circle, draw, edge label={node[midway,right,font=\scriptsize]{F}}
			[X, circle, draw, edge label={node[midway,right,font=\scriptsize]{F}}
			]
		]
	]
	\end{forest}
	\label{redundantB}}
	\hspace*{0.25\textwidth}
	\subfloat[][]{
	\centering
	\begin{forest}
	[F, circle, draw, edge label={node[midway,right,font=\scriptsize]{F}}
		[T, circle, draw, edge label={node[midway,right,font=\scriptsize]{F}}
			[X, circle, draw, edge label={node[midway,right,font=\scriptsize]{F}}
			]
		]
	]
	\end{forest}
	\label{redundantC}}
	\captionof{figure}{In Figure \ref{redundantA} we see the tree, with \textit{X} denoting that the comparison has not be performed, in Figure \ref{redundantB} 
  we see
  the case that the first comparison results in true. Since the first
  comparison resulted in true the next two are never performed, and they can be
  forgotten. The same idea can be seen in Figure \ref{redundantC} where the first results is
  false, the second in true, and so the last is never performed.}
	\label{redundant}
\end{figure}

One thing that is also important is that we are tuning with multiple datasets.
Therefore we can not just think of all the unique paths through one tree, instead we
need to think of each dataset as having it's own tree with it's own 
\texttt{comparison\_value} for each threshold. This means the search space is
not every unique path through the tree but every possible unique combination
of paths through the many trees we now have.
A visual example is show, and explained, in Figure \ref{combination}. 
\forestset{
  colour my roots/.style={
    before typesetting nodes={
      edge=#1,
      for ancestors={
        edge=#1,
        #1,
      },
      #1,
    }
  },
  my edge label/.style={
    edge label={
      node [midway, fill=white, font=\scriptsize] {#1}
    }
  }
}

\begin{figure}[h]
\centering
	\subfloat[][small]{
	\centering
	\begin{forest}
[Th4
	[V1,edge label={node[midway,left,font=\scriptsize]{T}}]
	[Th5,edge label={node[midway,right,font=\scriptsize]{F}}
		[V2,edge label={node[midway,left,font=\scriptsize]{T}}]
		[Th6,edge label={node[midway,right,font=\scriptsize]{F}}
			[V3,colour my roots=red, edge label={node[midway,left,font=\scriptsize]{T}}]
			[Th7,edge label={node[midway,right,font=\scriptsize]{F}}
				[Th16,edge label={node[midway,left,font=\scriptsize]{F}}
					[V5,edge label={node[midway,left,font=\scriptsize]{T}}]
					[Th17,edge label={node[midway,right,font=\scriptsize]{F}}
						[V6, edge label={node[midway,left,font=\scriptsize]{T}}]
						[V7,edge label={node[midway,right,font=\scriptsize]{F}}]
					]
				]
				[Th8,edge label={node[midway,left,font=\scriptsize]{F}}
					[V8,edge label={node[midway,left,font=\scriptsize]{T}}]
					[Th9,edge label={node[midway,right,font=\scriptsize]{F}}
						[V9, edge label={node[midway,left,font=\scriptsize]{T}}]
						[V10,edge label={node[midway,right,font=\scriptsize]{F}}]
					]
				]
			[V4, edge label={node[midway, right,font=\scriptsize]{T}}]
			]
		]
	]	
]
\end{forest}
	\label{combinationA}}
	\subfloat[][medium]{
	\centering
		\begin{forest}
[Th4
	[V1,edge label={node[midway,left,font=\scriptsize]{T}}]
	[Th5,edge label={node[midway,right,font=\scriptsize]{F}}
		[V2,edge label={node[midway,left,font=\scriptsize]{T}}]
		[Th6,edge label={node[midway,right,font=\scriptsize]{F}}
			[V3,edge label={node[midway,left,font=\scriptsize]{T}}]
			[Th7,edge label={node[midway,right,font=\scriptsize]{F}}
				[Th16,edge label={node[midway,left,font=\scriptsize]{F}}
					[V5,edge label={node[midway,left,font=\scriptsize]{T}}]
					[Th17,edge label={node[midway,right,font=\scriptsize]{F}}
						[V6, edge label={node[midway,left,font=\scriptsize]{T}}]
						[V7,edge label={node[midway,right,font=\scriptsize]{F}}]
					]
				]
				[Th8,edge label={node[midway,left,font=\scriptsize]{F}}
					[V8,edge label={node[midway,left,font=\scriptsize]{T}}]
					[Th9,edge label={node[midway,right,font=\scriptsize]{F}}
						[V9, edge label={node[midway,left,font=\scriptsize]{T}}]
						[V10,edge label={node[midway,right,font=\scriptsize]{F}}]
					]
				]
			[V4,colour my roots=green,edge label={node[midway, right,font=\scriptsize]{T}}]
			]
		]
	]	
]
\end{forest}
	\label{combinationB}}
	\subfloat[][large]{
	\centering
	\begin{forest}
[Th4
	[V1,edge label={node[midway,left,font=\scriptsize]{T}}]
	[Th5,edge label={node[midway,right,font=\scriptsize]{F}}
		[V2,edge label={node[midway,left,font=\scriptsize]{T}}]
		[Th6,edge label={node[midway,right,font=\scriptsize]{F}}
			[V3,edge label={node[midway,left,font=\scriptsize]{T}}]
			[Th7,edge label={node[midway,right,font=\scriptsize]{F}}
				[Th16,edge label={node[midway,left,font=\scriptsize]{F}}
					[V5,edge label={node[midway,left,font=\scriptsize]{T}}]
					[Th17, edge label={node[midway,right,font=\scriptsize]{F}}
						[V6, colour my roots=blue, edge label={node[midway,left,font=\scriptsize]{T}}]
						[V7, edge label={node[midway,right,font=\scriptsize]{F}}]
					]
				]
				[Th8,edge label={node[midway,left,font=\scriptsize]{F}}
					[V8,edge label={node[midway,left,font=\scriptsize]{T}}]
					[Th9,edge label={node[midway,right,font=\scriptsize]{F}}
						[V9 ,colour my roots=blue, edge label={node[midway,left,font=\scriptsize]{T}}]
						[V10, edge label={node[midway,right,font=\scriptsize]{F}}]
					]
				]
			[V4,edge label={node[midway, right,font=\scriptsize]{T}}]
			]
		]
	]	
]
\end{forest}
	\label{combinationC}}
	\caption{We see an example of how three dataset might choose different paths
    through the tree depending on their size, with the same set of threshold
    parameters. The general idea is that the
    longer down we go in the tree the more parallelism is utilized. The three trees here show a unique path through each, and this combination of these three paths, is also unique. We need to check all of such unique combinations. This tree is
    based on the \textit{LocVolCalb} program. The paths show here are not necessarily the
    combination of paths any dataset would take based on any combination of
    threshold values.}
	\label{combination}
\end{figure}


\subsection{Identifying all unique combinations}
The values that the thresholds are compared to in each dataset are extracted
to a list of values for each threshold. Then they are sorted and duplicates are
removed. Since the comparisons are all less than or equal
(\texttt{Threshold\_value <= comparison\_value}) comparisons we can simply
set the threshold to the smallest \texttt{comparison\_value} and be sure that the comparison
will result in true for all our dataset. It's also possible to construct a
value that will guarantee that the comparison will result in false. We do this
by taking the largest value for each threshold, where all but the largest
dataset will result in false, and then incrementing that value by one. If we
then set the threshold to this new value the comparison will be false for the
largest dataset as well. We put this false-value at
the end of the list of values we extracted from the program. We are left with a
list of values for each threshold where the first value will make the comparison for
every dataset result in true, the last will make the comparison result in false
for every dataset, and the values between will result in true for some and
false for others. These lists are then put on their corresponding threshold in
the tree we've constructed.
\\
To explain how the paths are found we will use the example tree in Figure \ref{treeNoName0}.
\forestset{sn edges/.style={for tree={parent anchor=east, child anchor=west}}}

\begin{figure}[h]
\centering
\begin{forest}
	for tree={%
		grow=east,
		l sep=0.1cm,
		s sep=0.1cm,
		minimum height=0.8cm,
		minimum width=1cm,
	}
[\texttt{('nullroot', false, [])}
	[\texttt{('Th1', false, [p11, p12, p13])}
		[\texttt{('Th4', true, [p41, p42])}]
		[\texttt{('Th3', false, [p31, p32])}]
	]
	[\texttt{('Th0', false, [p01, p02, p03])}
		[\texttt{('Th2', false, [p21, p22, p23])}]
		[\texttt{('Th5', false, [p51, p52])}]
	]
]
\end{forest}
\caption{A tree representing the structure of the threshold parameters
that is more true to the actual implementation}
\label{treeNoName0}
\end{figure}

The Futhark compiler cannot currently create such a tree, but
it highlights some of the changes that might be implemented in the Futhark
compiler in the future, and shows that our autotuner will handle such cases. \\
We construct all possible combinations in a bottom up manner. First
all the paths possible in the leaves are found. This is of course trivial as
each combination only consists of a single threshold, so the result will just
be each value as a singleton list (see Figure \ref{treeNoName1}).
\forestset{sn edges/.style={for tree={parent anchor=east, child anchor=west}}}

\begin{figure}[h]
\centering
\begin{forest}
[, phantom
	[\texttt{('nullroot', false, [])}]
	[\texttt{('Th1', false, [p11, p12])}]
	[\texttt{('Th4', true, [p41, p42])}]
]
\end{forest}
\begin{forest}
[, phantom
	[\texttt{('Th3', false, [p31, p32])}]
	[\texttt{('Th0', false, [p01, p02, p03])}]
]
\end{forest}
\begin{forest}
[, phantom
	[\texttt{('Th2', false, [p21, p22, p23])}]
	[\texttt{('Th5', false, [p51, p52])}]
]
\end{forest}
\caption{Mathias er cute!}
\label{treeNoName1}
\end{figure}

Next we look at the parents of the leaves. Here we need to combine the nodes
own values with the children combinations to create every possible
combination. There are a two cases that can occur here. \\
The first case is if we only have a single child for a certain outcome of the
comparison, or we have two children that depend on the same results of the comparison. 
\forestset{sn edges/.style={for tree={parent anchor=east, child anchor=west}}}

\begin{figure}[h]
\centering
\subfloat[][]{
\begin{forest}
	for tree={%
		grow=east,
		l sep=0.5cm,
		s sep=0.5cm,
		minimum height=0.8cm,
		minimum width=1cm,
	}
[, phantom
	[\texttt{('Th1', false, [p11, p12, p13])}
		[\texttt{('Th4', true, [p41, p42])}]
  [\texttt{('Th3', false, [p31, p32])}]]
]
\end{forest}
\label{treeNoName20}}
\\
\subfloat[][]{
	\begin{forest}
		for tree={%
			grow=east,
			l sep=0.5cm,
			s sep=0.5cm,
			minimum height=0.8cm,
			minimum width=1cm,
		}
		[\texttt{('Th0', false, [p01, p02, p03])}
		[\texttt{('Th2', false, [p21, p22, p23])}]
		[\texttt{('Th5', false, [p51, p52])}]
		]
	\end{forest}
	\label{treeNoName21}}
  \caption{The two subtrees of the root of the tree in figure \ref{treeNoName0}}
\label{treeNoName2}
\end{figure}

In Figure \ref{treeNoName20} we have an example of the first case. Here we take the first
value of the node, will always result in true, and combine it with all
the combinations we got from the true child. The same is done for the last value of
the node, the one that will always have the comparison result in false, and combine it with
the combinations from the false child. Then we take the values in the middle
and combine them with all the combinations of both children, as these result in
the comparison being true or false depending on the dataset. If there was only
one child the part that was explained about the other is not performed and
instead the true/false only value, whichever type of child did not exist, is
put in a singleton list as the result of a comparison with it  will lead to a
code version.\\
The example in figure \ref{treeNoName20} would give us the following result.\vspace{1em}

{\centering
  \texttt{[[('TH0',p01)],[('TH0',p11),('TH3',p31),('TH4',p41)],}\\
  \texttt{[('TH0',p11),('TH3',p31),('TH4',p42)],[('TH0',p11),('TH3',p32),('TH4',p41)],}\\
  \texttt{[('TH0',p11),('TH3',p32),('TH4',p42)],[('TH0',p12),('TH3',p31),('TH4',p41)],}\\
  \texttt{[('TH0',p12),('TH3',p31),('TH4',p42)],[('TH0',p12),('TH3',p32),('TH4',p41)],}\\
  \hspace{10em}\texttt{[('TH0',p12),('TH3',p32),('TH4',p42)]]}
}
\vspace{1em}

The second case is when 
there are multiple children that depend on the comparison result, as in
\ref{treeNoName21}. In this case we need to combine the combinations of
threshold value of all these children into all combinations of those. These cases
occur when the program can end in multiple code versions which means we need to check
every combination of those code versions. After combining the results from the
children we create every combination of the childrens, now combined, results
and the nodes own values. \\
The example in figure \ref{treeNoName21} would give us the following result.\vspace{1em}

{\centering
  \texttt{[[('TH1',p11),('TH2',p21)],[('TH1',p11),('TH2',p22)],[('TH1',p11),('TH2',p23)],}\\
  \texttt{[[('TH1',p12),('TH2',p21)],[('TH1',p12),('TH2',p22)],[('TH1',p12),('TH2',p23)],}\\
  \texttt{[[('TH1',p12),('TH5',p51)],[('TH1',p12),('TH5',p52)],}\\
  \texttt{[[('TH1',p13),('TH5',p51)],[('TH1',p13),('TH5',p52)]]}\\
}


\subsection{Handling Loops}
The loop construct that Futhark\footnote{Details regarding loops in Futhark, can be found in the documentation \url{https://futhark.readthedocs.io/en/latest/language-reference.html?highlight=loop}} has will run a piece of code several times with
a value that possibly changes each iteration. This means that, unlike the
assumption we made earlier, the threshold
parameters that correspond to the code inside a loop will be compared multiple
times, and that the value is compared to could change in each iteration. The
effect of this is that an exhaustive search cannot just assume that each
threshold is only compared a single time per dataset. This problem is already
trivially solved by the solution put forward in this section, since
these extra comparisons will just be added to the list of comparison values and
the different combinations they represent will be tried out.
