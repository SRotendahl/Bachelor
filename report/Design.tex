\section{Design \& Implementation}

%\subsection{Auto-tuning in Introduction}
%in broad terms auto-tuning is the process of automatically tuning a set of
%parameters so as to optimize some aspect of the program when run. This
%process can be done both at compile- or runtime, but in the case of Futhark,
%and this report it's done at runtime. In our, and most other, cases this aspect
%we are trying to optimize will be the runtime, that we are trying to minimize. 
Our goal when autotuning a Futhark program is to find the set of values for the
threshold parameters that will minimize the running time of the program on the
given datasets. There is no way to be completely sure
which combination of values will lead to the fastest running time on the given
datasets, except to try them all out. This is called an exhaustive search, and
it is the way we will find the optimal set of values for the threshold
parameters. 

As we saw in section \ref{BabyGotBack}, the threshold parameters within a program exhibit a tree structure (for example Figure \ref{LocVolCalibTree}), of dependencies. Therefore we will build a structure analogous to these 
\textbf{Either 'Building the Tree' and 'Limiting Search Space' need to be
swapped somehow, or we need a way to tie intro to 'Building the Tree'. I have tried to tie them together}

\subsection{Building the Tree}
Every, compiled, Futhark program can be given a flag so the thresholds, and their structure, is printed out. The syntax of the string representing each threshold parameter
can be seen in Figure \ref{thresholdSyntax}.
\begin{figure}[h]
	$$\underbrace{\texttt{main.suff\_outer\_par\_6}}_\text{threshold parameter \#6} \overbrace{\texttt{(threshold (!main.suff\_outer\_par\_4} \underbrace{\texttt{!main.suff\_intra\_par\_5))}}_\text{(!) indicates a false comparison}}^\text{list of thresholds that \#6 is dependent on}$$
	\caption{The syntax of thresholds, printed by Futhark.}
	\label{thresholdSyntax}
\end{figure}
We translate the list of these strings into a tree structure similar to the
trees we have see in Section \ref{BabyGotBack}. 
We start by finding the thresholds that does not depend on any other thresholds, 
%being true or false, in order for them to be evaluated, 
%this will make them root nodes in their respective trees. 
these thresholds will be root nodes of their respective trees.
Currently there will only exist one of these root nodes, but in order to make the program
adaptable to future changes to the compiler, we take into account, the possibility that the
structure of the thresholds will exhibit a forest. In order to process a forest, in the same way we would a tree, 
we create a root node that has no useful information, but will simply be the
parent of the forest, hence for referenced as the \textit{nullroot}. \\

To construct the rest of the tree we execute recursively on all the children
of the previous node, finding one who has the same dependencies as the current
node, but also depends on the current node itself. All of these are made
children of the current node, and the function is executed on each of the children. The
recursion ends when no children are found for a node.

Unlike the theoretical trees, seen previously, the implemented trees cannot carry information on
the edges themselves, this means the information is instead encoded in the
children. So each node is a tuple containing the name of the threshold
parameter and a boolean denoting the result of the parents threshold
comparison that lead to the node. The process of constructing the tree can be seen in Figure \ref{buildThresTree}.
\forestset{sn edges/.style={for tree={parent anchor=south, child anchor=north}}}

\begin{figure}[h]
\centering
	\begin{minipage}{0.25\textwidth}
	\centering
	\begin{forest}
	[\texttt{(NULL, false)}]
	\end{forest}
	\end{minipage}\hspace*{\fill}
	\begin{minipage}{0.25\textwidth}
	\centering
	\begin{forest}
	[\texttt{(NULL, false)}
		[\texttt{(Th4, false)}]
	]
	\end{forest}
	\end{minipage}\hspace*{\fill}
	\begin{minipage}{0.25\textwidth}
	\centering
	\begin{forest}
	[\texttt{(NULL, false)}
		[\texttt{(Th4, false)}
			[\texttt{(Th5, false)}
			]
		]
	]
	\end{forest}
	\end{minipage}\hspace*{\fill}
	\begin{minipage}{0.25\textwidth}
	\centering
	\begin{forest}
	[\texttt{(NULL, false)}
		[\texttt{(Th4, false)}
			[\texttt{(Th5, false)}
				[\texttt{(Th6, false)}]
			]
		]
	]
	\end{forest}
	\end{minipage}
	\captionof{figure}{The recusive steps of building the tree structure for the thresholds, with (Th) being a threshold parameter. The list of thresholds for the tree would be \texttt{[main.suff\_intra\_par\_5 (threshold (!main.suff\_outer\_par\_4)), main.suff\_outer\_par\_4 (threshold ()), main.suff\_outer\_par\_6 (threshold (!main.suff\_outer\_par\_4 !main.suff\_intra\_par\_5))]}}
	%We might want to remove the list of thresholds
	\label{buildThresTree}
\end{figure}


While this solution works currently, a change in the Futhark compiler could
result in nodes having multiple parents, and therefore requiring more
information than a single boolean. This problem can be solved quite simply by
duplicating the problematic subtree, thereby having a version of the subtree that came from a
comparisons being true, and one for comparisons being false.

\subsection{Limiting Search Space}
If we knew nothing about
the thresholds, or their structure we would have to try every configuration of
numbers between 0 and whatever the highest number their implementation allows. 
With a simple program such a matrix-matrix multiplication (which has 4 thresholds, see Figure \ref{MatMultTreeFilled}), and assuming an upper limit of $2^{15}$ which is the threshold default, and therefore a conservative limit,
we would end up with $\left(2^{15}\right)^4 \approx 1.153\times10^{18}$ configurations.
This search would obviously take far longer than any of us would be alive, so we need a way to limit the search space.
\\
They way we limit the search space is by using our knowledge of the threshold
parameters and only trying values that would result in a different sets of code 
version being executed on at least one of the datasets. More specifically, we will
use the knowledge that the comparisons are of the form
(\texttt{Threshold\_value <= comparison\_value}). Since the
\texttt{comparison\_value} is constant for a certain dataset there are effectively
only two different threshold values that need to be checked for one threshold 
in isolation. One where it's equal to the \texttt{comparison\_value} and one
where it's $\texttt{comparison\_value} + 1$, as these will result in true and
false respectively, with those being the only options.
Of course the thresholds are not in isolation, but structured as we've seen
Figure \ref{LocVolCalibTree} for example, so some comparison result in earlier 
thresholds will make later thresholds in the tree redundant, as they are never compared. We can see this idea in Figure \ref{redundant}. A concrete example would be Figure \ref{MatMultTreeFilled}, with the thresholds $t_i = [10, 20, 30, 40]$, and the matrix sizes $n=20, p=30, n\times p=600, m=50$, so the predicate of $t_0 \leq n$ would result in taking the \texttt{true} branch, however this is also the case for $t_0 = 11$, making these two different configurations, having the same dynamic behavior.
\forestset{sn edges/.style={for tree={parent anchor=south, child anchor=north}}}

\begin{figure}[h]
\centering
	\subfloat[][]{
	\centering
	\begin{forest}
	[$\quad$, circle, draw, edge label={node[midway,right,font=\scriptsize]{F}}
		[$\quad$, circle, draw, edge label={node[midway,right,font=\scriptsize]{F}}
			[$\quad$, circle, draw, edge label={node[midway,right,font=\scriptsize]{F}}
			]
		]
	]
	\end{forest}
	\label{redundantA}}
	\hspace*{0.25\textwidth}
	\subfloat[][]{
	\centering
	\begin{forest}
	[T, circle, draw, edge label={node[midway,right,font=\scriptsize]{F}}
		[X, circle, draw, edge label={node[midway,right,font=\scriptsize]{F}}
			[X, circle, draw, edge label={node[midway,right,font=\scriptsize]{F}}
			]
		]
	]
	\end{forest}
	\label{redundantB}}
	\hspace*{0.25\textwidth}
	\subfloat[][]{
	\centering
	\begin{forest}
	[F, circle, draw, edge label={node[midway,right,font=\scriptsize]{F}}
		[T, circle, draw, edge label={node[midway,right,font=\scriptsize]{F}}
			[X, circle, draw, edge label={node[midway,right,font=\scriptsize]{F}}
			]
		]
	]
	\end{forest}
	\label{redundantC}}
	\captionof{figure}{In Figure \ref{redundantA} we see the tree, with \textit{X} denoting that the comparison has not be performed, in Figure \ref{redundantB} 
  we see
  the case that the first comparison results in true. Since the first
  comparison resulted in true the next two are never performed, and they can be
  forgotten. The same idea can be seen in Figure \ref{redundantC} where the first results is
  false, the second in true, and so the last is never performed.}
	\label{redundant}
\end{figure}

We also limit our search space by removing duplicate thresholds comparisons. It is possible for different datasets to have the same comparison value for a threshold, this will give duplicates. We remove these possible duplicates to limit the search space for redundant configurations.
One thing that is also important is that we are tuning with multiple datasets,
so we can not just think of all the unique paths through one tree. Instead we
need to think of each dataset as having it's own tree, with it's own 
\texttt{comparison\_value} for each threshold. This means the search space is
not every unique path through the tree, but every possible unique combination
of paths through the many trees we now have. A visual example is show, and explained in Figure \ref{combination}. 
\forestset{
  colour my roots/.style={
    before typesetting nodes={
      edge=#1,
      for ancestors={
        edge=#1,
        #1,
      },
      #1,
    }
  },
  my edge label/.style={
    edge label={
      node [midway, fill=white, font=\scriptsize] {#1}
    }
  }
}

\begin{figure}[h]
\centering
	\subfloat[][small]{
	\centering
	\begin{forest}
[Th4
	[V1,edge label={node[midway,left,font=\scriptsize]{T}}]
	[Th5,edge label={node[midway,right,font=\scriptsize]{F}}
		[V2,edge label={node[midway,left,font=\scriptsize]{T}}]
		[Th6,edge label={node[midway,right,font=\scriptsize]{F}}
			[V3,colour my roots=red, edge label={node[midway,left,font=\scriptsize]{T}}]
			[Th7,edge label={node[midway,right,font=\scriptsize]{F}}
				[Th16,edge label={node[midway,left,font=\scriptsize]{F}}
					[V5,edge label={node[midway,left,font=\scriptsize]{T}}]
					[Th17,edge label={node[midway,right,font=\scriptsize]{F}}
						[V6, edge label={node[midway,left,font=\scriptsize]{T}}]
						[V7,edge label={node[midway,right,font=\scriptsize]{F}}]
					]
				]
				[Th8,edge label={node[midway,left,font=\scriptsize]{F}}
					[V8,edge label={node[midway,left,font=\scriptsize]{T}}]
					[Th9,edge label={node[midway,right,font=\scriptsize]{F}}
						[V9, edge label={node[midway,left,font=\scriptsize]{T}}]
						[V10,edge label={node[midway,right,font=\scriptsize]{F}}]
					]
				]
			[V4, edge label={node[midway, right,font=\scriptsize]{T}}]
			]
		]
	]	
]
\end{forest}
	\label{combinationA}}
	\subfloat[][medium]{
	\centering
		\begin{forest}
[Th4
	[V1,edge label={node[midway,left,font=\scriptsize]{T}}]
	[Th5,edge label={node[midway,right,font=\scriptsize]{F}}
		[V2,edge label={node[midway,left,font=\scriptsize]{T}}]
		[Th6,edge label={node[midway,right,font=\scriptsize]{F}}
			[V3,edge label={node[midway,left,font=\scriptsize]{T}}]
			[Th7,edge label={node[midway,right,font=\scriptsize]{F}}
				[Th16,edge label={node[midway,left,font=\scriptsize]{F}}
					[V5,edge label={node[midway,left,font=\scriptsize]{T}}]
					[Th17,edge label={node[midway,right,font=\scriptsize]{F}}
						[V6, edge label={node[midway,left,font=\scriptsize]{T}}]
						[V7,edge label={node[midway,right,font=\scriptsize]{F}}]
					]
				]
				[Th8,edge label={node[midway,left,font=\scriptsize]{F}}
					[V8,edge label={node[midway,left,font=\scriptsize]{T}}]
					[Th9,edge label={node[midway,right,font=\scriptsize]{F}}
						[V9, edge label={node[midway,left,font=\scriptsize]{T}}]
						[V10,edge label={node[midway,right,font=\scriptsize]{F}}]
					]
				]
			[V4,colour my roots=green,edge label={node[midway, right,font=\scriptsize]{T}}]
			]
		]
	]	
]
\end{forest}
	\label{combinationB}}
	\subfloat[][large]{
	\centering
	\begin{forest}
[Th4
	[V1,edge label={node[midway,left,font=\scriptsize]{T}}]
	[Th5,edge label={node[midway,right,font=\scriptsize]{F}}
		[V2,edge label={node[midway,left,font=\scriptsize]{T}}]
		[Th6,edge label={node[midway,right,font=\scriptsize]{F}}
			[V3,edge label={node[midway,left,font=\scriptsize]{T}}]
			[Th7,edge label={node[midway,right,font=\scriptsize]{F}}
				[Th16,edge label={node[midway,left,font=\scriptsize]{F}}
					[V5,edge label={node[midway,left,font=\scriptsize]{T}}]
					[Th17, edge label={node[midway,right,font=\scriptsize]{F}}
						[V6, colour my roots=blue, edge label={node[midway,left,font=\scriptsize]{T}}]
						[V7, edge label={node[midway,right,font=\scriptsize]{F}}]
					]
				]
				[Th8,edge label={node[midway,left,font=\scriptsize]{F}}
					[V8,edge label={node[midway,left,font=\scriptsize]{T}}]
					[Th9,edge label={node[midway,right,font=\scriptsize]{F}}
						[V9 ,colour my roots=blue, edge label={node[midway,left,font=\scriptsize]{T}}]
						[V10, edge label={node[midway,right,font=\scriptsize]{F}}]
					]
				]
			[V4,edge label={node[midway, right,font=\scriptsize]{T}}]
			]
		]
	]	
]
\end{forest}
	\label{combinationC}}
	\caption{Text}
	\label{combination}
\end{figure}

\subsection{Identifying all unique combinations}
The values that the thresholds are compared to in each dataset are extracted,
to a list of values for each threshold. Then they are sorted and duplicates are
removed. Since the comparisons are all less than or equal
(\texttt{Threshold\_value <= comparison\_value}) comparisons we can simply
set the threshold to the smallest \texttt{comparison\_value} and be sure that the comparison
will result in true for all our dataset. It's also possible to construct a
value that will guarantee that the comparison will result in false. We do this
by taking the largest value for each threshold, where all but the largest
dataset will result in false, and then incrementing that value by one. If we then set the threshold to that value, then the comparison will be false for the largest dataset, as well as all the others. We put this false-value at
the end of the list of values we extracted from the program. We are left with a
list of values for each threshold, the first value will make the comparison for
every dataset result in true, the last will make the comparison result in false
for every dataset, and the values between will result in true for some, and
false for others. These values are then put on their corresponding threshold in
the tree we've constructed.
\\
To explain how the paths are found we will use the example tree in Figure \ref{treeNoName0}.
\forestset{sn edges/.style={for tree={parent anchor=east, child anchor=west}}}

\begin{figure}[h]
\centering
\begin{forest}
	for tree={%
		grow=east,
		l sep=0.1cm,
		s sep=0.1cm,
		minimum height=0.8cm,
		minimum width=1cm,
	}
[\texttt{('nullroot', false, [])}
	[\texttt{('Th1', false, [p11, p12, p13])}
		[\texttt{('Th4', true, [p41, p42])}]
		[\texttt{('Th3', false, [p31, p32])}]
	]
	[\texttt{('Th0', false, [p01, p02, p03])}
		[\texttt{('Th2', false, [p21, p22, p23])}]
		[\texttt{('Th5', false, [p51, p52])}]
	]
]
\end{forest}
\caption{A tree representing the structure of the threshold parameters
that is more true to the actual implementation}
\label{treeNoName0}
\end{figure}

The Futhark compiler cannot currently create such a tree, but
it highlights some of the changes that  might be implemented in the Futhark
compiler in the future, and shows that our autotuner will handle such cases. \\
We construct all possible combinations in a bottom up manner. First
all the paths possible in the leaves are found, this is of course trivial as
each combination only consists of a single threshold, so the result will just
be each value as a singleton list (see Figure \ref{treeNoName1}).
\forestset{sn edges/.style={for tree={parent anchor=east, child anchor=west}}}

\begin{figure}[h]
\centering
\begin{forest}
[, phantom
	[\texttt{('nullroot', false, [])}]
	[\texttt{('Th1', false, [p11, p12])}]
	[\texttt{('Th4', true, [p41, p42])}]
]
\end{forest}
\begin{forest}
[, phantom
	[\texttt{('Th3', false, [p31, p32])}]
	[\texttt{('Th0', false, [p01, p02, p03])}]
]
\end{forest}
\begin{forest}
[, phantom
	[\texttt{('Th2', false, [p21, p22, p23])}]
	[\texttt{('Th5', false, [p51, p52])}]
]
\end{forest}
\caption{Mathias er cute!}
\label{treeNoName1}
\end{figure}

\begin{figure}[]
  \centering
  \textbf{Jeg kan ikke finde ud af at få det sammen med træerne}
$\to \texttt{[[('TH0',p01)],
[('TH0',p11),('TH3',p31),('TH4',p41)],[('TH0',p11),('TH3',p31),('TH4',p42)],[('TH0',p11),('TH3',p32),('TH4',p41)],[('TH0',p11),('TH3',p32),('TH4',p42)]
[('TH0',p12),('TH3',p31),('TH4',p41)],[('TH0',p12),('TH3',p31),('TH4',p42)],[('TH0',p12),('TH3',p32),('TH4',p41)],[('TH0',p12),('TH3',p32),('TH4',p42)]
]}$

$\to \texttt{[[('TH1',p11),('TH2',p21)],[('TH1',p11),('TH2',p22)],[('TH1',p11),('TH2',p23)]
             [[('TH1',p12),('TH2',p21)],[('TH1',p12),('TH2',p22)],[('TH1',p12),('TH2',p23)]
             [[('TH1',p12),('TH5',p51)],[('TH1',p12),('TH5',p52)]
             [[('TH1',p13),('TH5',p51)],[('TH1',p13),('TH5',p52)]]}$
\end{figure}
Next we look at the parents of the leaves. Here we need to combine the nodes
own values with the children combinations to create every possible
combination. There are a two cases that can occur here. \\
The first case is if we only have a single child, which will dependent on the result of
the comparison being true or false, or we have two children that depend on
opposite results of the comparison. 
\forestset{sn edges/.style={for tree={parent anchor=east, child anchor=west}}}

\begin{figure}[h]
\centering
\subfloat[][]{
\begin{forest}
	for tree={%
		grow=east,
		l sep=0.5cm,
		s sep=0.5cm,
		minimum height=0.8cm,
		minimum width=1cm,
	}
[, phantom
	[\texttt{('Th1', false, [p11, p12, p13])}
		[\texttt{('Th4', true, [p41, p42])}]
  [\texttt{('Th3', false, [p31, p32])}]]
]
\end{forest}
\label{treeNoName20}}
\\
\subfloat[][]{
	\begin{forest}
		for tree={%
			grow=east,
			l sep=0.5cm,
			s sep=0.5cm,
			minimum height=0.8cm,
			minimum width=1cm,
		}
		[\texttt{('Th0', false, [p01, p02, p03])}
		[\texttt{('Th2', false, [p21, p22, p23])}]
		[\texttt{('Th5', false, [p51, p52])}]
		]
	\end{forest}
	\label{treeNoName21}}
\caption{Mathias er cute!}
\label{treeNoName2}
\end{figure}

In Figure \ref{treeNoName20} we have an example of the first case. Here we take the first
value of the node, the one that will always be true, and combine it with all
the combinations we got from the leaf. The same is done for the last value of
the node, the one that will always have the comparison result in false, and combine it with
the combinations from the false child. Then we take the values in the middle
and combine them with all the combinations of both children, as these result in
the comparison being true or false depending on the dataset. If there was only
one child the part that was explained about the other is not performed and
instead the true/false only value, whichever type of child did not exist, is
put in a singleton list as that value will lead to a code version.\\
The example in figure \ref{treeNoName20} would give us the following result.\vspace{1em}

{\centering
  \texttt{[[('TH0',p01)],[('TH0',p11),('TH3',p31),('TH4',p41)],}\\
  \texttt{[('TH0',p11),('TH3',p31),('TH4',p42)],[('TH0',p11),('TH3',p32),('TH4',p41)],}\\
  \texttt{[('TH0',p11),('TH3',p32),('TH4',p42)],[('TH0',p12),('TH3',p31),('TH4',p41)],}\\
  \texttt{[('TH0',p12),('TH3',p31),('TH4',p42)],[('TH0',p12),('TH3',p32),('TH4',p41)],}\\
  \hspace{10em}\texttt{[('TH0',p12),('TH3',p32),('TH4',p42)]]}
}
\vspace{1em}

The second case is when 
there are multiple children that depend on the comparison result, as in
\ref{treeNoName21}. In this case we need to combine the threshold value
combinations of all these children into all combinations of those. These cases
occur when the program can end in multiple code versions, so we need to check
every combination of those code versions. After combining the results from the
children we create every combination of the children's, now combined, results
and the nodes own values. \\
The example in figure \ref{treeNoName21} would give us the following result.\vspace{1em}

{\centering
  \texttt{[[('TH1',p11),('TH2',p21)],[('TH1',p11),('TH2',p22)],[('TH1',p11),('TH2',p23)],}\\
  \texttt{[[('TH1',p12),('TH2',p21)],[('TH1',p12),('TH2',p22)],[('TH1',p12),('TH2',p23)],}\\
  \texttt{[[('TH1',p12),('TH5',p51)],[('TH1',p12),('TH5',p52)],}\\
  \texttt{[[('TH1',p13),('TH5',p51)],[('TH1',p13),('TH5',p52)]]}\\
}


\subsection{Handling Loops}
The loop construct that Futhark\textbf{REF} has will run a piece of code several times with
a value that possibly changes each iteration. This means that the threshold
parameters that correspond to the code inside a loop will be compared multiple
times, and that the value is compared to could change in each iteration. The
effect of this is that an exhaustive search cannot just assume that each
threshold is only compared a single time per dataset. This problem is
trivially solved by the solution put forward earlier in this section, since
these extra comparisons will just be added to the list of comparison values and
the different combinations they represent will be tried out.
