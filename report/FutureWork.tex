\section{Reflections \& Future work (Very unfinished)}

Our autotuner does what it's supposed to, performs an exhaustive search of all
possible combinations of code versions in the given datasets and returns the
one that performs the best. This means that in most cases the results it gives
are better than the current autotuner. The problem it has is that to perform an
exhaustive search and the number of combinations that needs to be checked increases
very quickly as you add more datasets. In the \texttt{LocVolCalib} program the
number of combinations that needs to be checked for one dataset is 13, while
the number of combinations that needs to be checked when tune for all
three datasets is 5463. 

An oversight we made in our implementation was to not stop an execution, if it was slower than a previous one, so if we had a path $e_0$ which had a running time of $10$ seconds, and we were actively checking path $e_1$, then we should stop the execution of $e_1$ if it exceeds $10$ seconds. This is a fairly easy optimization to make and could give a big speedup. 
%vi kunne skrive videre på dette med noget execution order og worst og best case.   

\subsection{Synthetic data}
It's clear that our autotuner is fast if the number of datasets are low, so a
possible solution is to not use very many. The problem with not using many
datasets is that the parameters that are achieved from tuning might not
generalize well to new datasets, as their sizes might be totally different. So
instead of just leaving out datasets. The possible solution we propose is that
it might be possible to create 'synthetic' datasets that, when tuned on,
results in threshold values that achieves a good performance on all datasets of
interest. 
This idea can be somewhat seen in our results on \texttt{LocVolCalib} (See \ref{LocVolCalibAll})
where training only on the \textit{medium} training  dataset resulted in good
performances on \textit{medium} and \textit{small} test datasets. This training
did not result in a good performance for the \textit{large} test dataset, but
this says more about the \textit{medium} training dataset is not a good
synthetic dataset, and less that the whole idea wouldn't work.

\begin{itemize}
	\item \textbf{Further filtration} should be performed. There are heuristics
    about Futhark programs, that can be used to reduces the number, a specific
    one is nest of two \texttt{map}s. It is often that we want to work on
    nested data, like a matrix, and mapping over its columns and rows. For this
    we would use two \texttt{map}s, and very rarely is only the outer
    \texttt{map} executed in parallel. However our autotuner does not take such
    heuristics into account (otherwise it would not be an exhaustive search),
    but if heuristics such as these are implemented, the number of thresholds
    for \texttt{map-map} would be reduced exponentially with the number of
    datasets.
\end{itemize}

